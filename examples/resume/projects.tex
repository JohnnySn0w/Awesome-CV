%-------------------------------------------------------------------------------
%	SECTION TITLE
%-------------------------------------------------------------------------------
\cvsection{Projects}


%-------------------------------------------------------------------------------
%	CONTENT
%-------------------------------------------------------------------------------
\begin{cventries}

%---------------------------------------------------------
\cventry
{Echo} % Proj name
{Local-compute-only AI voice asssitant} % summary
{\href{https://github.com/JohnnySn0w/Echo}{GitHub}} % Location
{Feb. 2024 - Current} % Date(s)
{
  \begin{cvitems} % Description(s) bullet points
    \item {Running an AI stack using only local compute.}
    \item {I can talk to my computer as-is, with no network traffic, and it talks back via high fidelity voice synthesis, and the sum of human knowledge.}
    \item {Current goals include adding RAG for induction of personal notes into AI knowledge base.}
  \end{cvitems}
}

%---------------------------------------------------------
\cventry
{MoD} % Proj name
{Using Node.js to implement a Multi-User Dungeon on Discord} % summary
{\href{https://github.com/JohnnySn0w/MoD}{GitHub}} % Location
{Nov. 2019 - May 2020} % Date(s)
{
  \begin{cvitems} % Description(s) bullet points
    \item {A handmade MUD engine that interfaces with Discord via the JavaScript chatbot API, Discord.js.}
    \item {Uses Serverless DynamoDB as a backend.}
    \item {Highly modular, allows for live editing of items, rooms, entities, etc.}
  \end{cvitems}
}

%---------------------------------------------------------
\cventry
{ArXver} % Proj name
{Webscraping tweets for archival} % summary
{\href{https://github.com/JohnnySn0w/ArXver}{GitHub}} % Location
{Jan. 2024} % Date(s)
{
  \begin{cvitems} % Description(s) bullet points
    \item {Twitter archival tool for converting tweets into Markdown}
    \item {Pages automatically tagged such that a PKM, such as Obsidian, can ingest and use them immediately.}
    \item {Synthesis of Python web scraping, JSON parsing, Markdown templating, and HTML arrangement.}
  \end{cvitems}
}

%---------------------------------------------------------
\cventry
{88x31 Collager} % Proj name
{Art generation tool using Geocities banner archive} % summary
{\href{https://github.com/JohnnySn0w/8831-collager}{GitHub}} % Location
{Mar. 2024 - Apr. 2024} % Date(s)
{
  \begin{cvitems} % Description(s) bullet points
    \item {Generating banner-per-pixel replacement videos.}
    \item {Done in Python with supporting libraries like Pillow and ImageMagic.}
    \item {Utilizes Python's multiprocessing module including shared\_memory.}
  \end{cvitems}
}

%---------------------------------------------------------
\end{cventries}
